% by Sean McKenna

\chapter{May 3\super{rd}, 2011}

\section{D\sub{6h} Character Representations}
\begin{center}
\begin{tabular}{l||r|r|r|r|r|r|r|r|r|r|r|r|}
  & E & 2C\sub{6} & 2C\sub{3} & C\sub{2} & 3C\sub{2}\super{'} & 3C\sub{2}\super{''} & i & 2S\sub{3} & 2S\sub{6} & $\sigma$\sub{h} & 3$\sigma$\sub{d} & 3$\sigma$\sub{v} \\ \hline \hline
  A\sub{1g} & 1 & 1 & 1 & 1 & 1 & 1 & 1 & 1 & 1 & 1 & 1 & 1 \\ \hline
  A\sub{2g} & 1 & 1 & 1 & 1 & -1 & -1 & 1 & 1 & 1 & 1 & -1 & -1 \\ \hline
  B\sub{1g} & 1 & -1 & 1 & -1 & 1 & -1 & 1 & -1 & 1 & -1 & 1 & -1 \\ \hline
  B\sub{2g} & 1 & -1 & 1 & -1 & -1 & 1 & 1 & -1 & 1 & -1 & -1 & 1 \\ \hline
  E\sub{1g} & 2 & 1 & -1 & -2 & 0 & 0 & 2 & 1 & -1 & -2 & 0 & 0 \\ \hline
  E\sub{2g} & 2 & -1 & -1 & 2 & 0 & 0 & 2 & -1 & -1 & 2 & 0 & 0 \\ \hline
  A\sub{1u} & 1 & 1 & 1 & 1 & 1 & 1 & -1 & -1 & -1 & -1 & -1 & -1 \\ \hline
  A\sub{2u} & 1 & 1 & 1 & 1 & -1 & -1 & -1 & -1 & -1 & -1 & 1 & 1 \\ \hline
  B\sub{1u} & 1 & -1 & 1 & -1 & 1 & -1 & -1 & 1 & -1 & 1 & -1 & 1 \\ \hline
  B\sub{2u} & 1 & -1 & 1 & -1 & -1 & 1 & -1 & 1 & -1 & 1 & 1 & -1 \\ \hline
  E\sub{1u} & 2 & 1 & -1 & -2 & 0 & 0 & -2 & -1 & 1 & 2 & 0 & 0 \\ \hline
  E\sub{2u} & 2 & -1 & -1 & 2 & 0 & 0 & -2 & 1 & 1 & -2 & 0 & 0 \\
  \hline
\end{tabular}
\end{center}

\begin{enumerate}
  \item This product is simply B\sub{1g}. \\ \\
  \begin{tabular}{l||r|r|r|r|r|r|r|r|r|r|r|r|}
    & E & 2C\sub{6} & 2C\sub{3} & C\sub{2} & 3C\sub{2}\super{'} & 3C\sub{2}\super{''} & i & 2S\sub{3} & 2S\sub{6} & $\sigma$\sub{h} & 3$\sigma$\sub{d} & 3$\sigma$\sub{v} \\ \hline \hline
    A\sub{1g} $\times$ B\sub{1g} & 1 & -1 & 1 & -1 & 1 & -1 & 1 & -1 & 1 & -1 & 1 & -1 \\
    \hline
  \end{tabular} \\

  \item This product is E\sub{2u}. \\ \\
  \begin{tabular}{l||r|r|r|r|r|r|r|r|r|r|r|r|}
    & E & 2C\sub{6} & 2C\sub{3} & C\sub{2} & 3C\sub{2}\super{'} & 3C\sub{2}\super{''} & i & 2S\sub{3} & 2S\sub{6} & $\sigma$\sub{h} & 3$\sigma$\sub{d} & 3$\sigma$\sub{v} \\ \hline \hline
    B\sub{2u} $\times$ E\sub{1g} & 2 & -1 & -1 & 2 & 0 & 0 & -2 & 1 & 1 & -2 & 0 & 0 \\
    \hline
  \end{tabular} \\

  \item This product is a superposition of the following states: B\sub{1g}, B\sub{2g}, and E\sub{1g}. Originally, the product is in a reduceable state with h = 24. Taking $1/24$ of each piece and multiplying those items by the proper amounts there yields the amount of each symmetric state. \\ \\
  \begin{tabular}{l||r|r|r|r|r|r|r|r|r|r|r|r|}
    & E & 2C\sub{6} & 2C\sub{3} & C\sub{2} & 3C\sub{2}\super{'} & 3C\sub{2}\super{''} & i & 2S\sub{3} & 2S\sub{6} & $\sigma$\sub{h} & 3$\sigma$\sub{d} & 3$\sigma$\sub{v} \\ \hline \hline
    E\sub{1g}$\times$B\sub{2g}$\times$A\sub{2u}$\times$E\sub{1u} & 4 & -1 & 1 & -4 & 0 & 0 & 4 & -1 & 1 & -4 & 0 & 0 \\ 
    \hline
  \end{tabular} \\
  \begin{center}
  \begin{tabular}{l||r|}
	\hline
    B\sub{1g} & 1 \\ \hline
    B\sub{2g} & 1 \\ \hline
    E\sub{1g} & 1 \\ \hline
    All the rest summed to... & 0 \\
    \hline
  \end{tabular}
  \end{center}
\end{enumerate}

\section{NO\sub{2} Symmetry Species}
For the molecular orbital defined by $\psi$\sub{MO} $= \psi$\sub{A}$ - \psi$\sub{B} where each wavefunction is defined by the 2p\sub{x} orbital on each of the oxygen atoms, it is true that the two orbitals are then anti-parallel and thus have more nodes (more energy typically) than the parallel arrangement. The chart for the point group C\sub{2v} is below. \\
\begin{center}
\begin{tabular}{l||r|r|r|r|}
  \hline
  & E & C\sub{2} & $\sigma$\sub{xz} & $\sigma$\sub{yz} \\ \hline \hline
  A\sub{1} & 1 & 1 & 1 & 1 \\ \hline
  A\sub{2} & 1 & 1 & -1 & -1 \\ \hline
  B\sub{1} & 1 & -1 & 1 & -1 \\ \hline
  B\sub{2} & 1 & -1 & -1 & 1 \\
  \hline
\end{tabular}
\end{center}
The orbitals obviously have E symmetry and C\sub{2} symmetry (simple rotation looks the same), but the orbitals look exactly opposite when any reflection in either plane is performed. That means the symmetry for this molecular orbital is A\sub{2}.

\section{Symmetry Type of PtCl\sub{4}$^{1-}$}
For this ion, we have a C\sub{4} axis. In addition, there are 4 C\sub{2} axes along the ion, all perpendicular to the principal axis, and there is a horizontal plane of symmetry. This means that PtCl\sub{4}$^{1-}$ ion has a point group symmetry of D\sub{4h}. \\ \\
Now, for the D\sub{4h} point group, the character table is a bit more complex. Obviously, the molecular orbital has E symmetry. Since the molecular orbitals are 3s orbitals with just opposite sign on the adjacent corners, we have 3 C\sub{2} axes of rotation. This means we must have an A or B symmetry type (no E). Some of the previous C\sub{2} axes result in opposite sign, like the axes that go through the edges of the square. This leaves us with some B form of symmetry. And since we have inversion symmetry as well of the square, we also know that we must have a g symmetry type, so either B\sub{1g} or B\sub{2g}. Lastly, since the reflection planes vertically must go through the atoms and not the edges of the square to keep symmetry, that means we are left with B\sub{2g} as the symmetry type of this ion's molecular orbital.


\chapter{May 5\super{th}, 2011}

\section{Structure \& Spectra}
For the three different structures, it is important to note that the ``smushed'' \ce{AB4} has similar geometry as the tetrahedral shape. While the bond lengths have flattened, they are not yet flat and do not yet have the ideal symmetry that a square planar molecule would have. The tetrahedral and ``smushed'' shapes should each have T\sub{d} point group symmetry, like that of methane. The square planar molecule will have the same point group as \ce{ICl4}: D\sub{4h}. \\ \\
While that may be useful to double-check an answer, it is not really needed. The key to this problem is the center of inversion, which only exists for a square planar shape, not for the other two. As the rule of mutual exclusion states, the molecule with an inversion center cannot have any overlap of peaks in the IR and Raman spectra. The center of inversion can be verified as it is also one of the symmetric elements of the D\sub{4h} point group. Thus, the fact that there is an overlap rules out the square planar shape leaving the tetrahedral and ``smushed'' shapes of \ce{AB4} as our two possible configurations.

\section{ChemActivity 22, Exercise 1}
The term symbol for both the 1$s^1$ and 2$s^1$ hydrogen atom is $^2s$. The term symbol for the 2$p^1$ is $^2p$. \\ \\
A transition from 1$s^1$ to 2$s^1$ is forbidden since there is no change in the orbital angular momentum ($\Delta L = 0$). This is the same case in the reverse direction as well, since $\Delta L = 0$ still. However, for a transition from 2$p^1$ to 1$s^1$ is allowed since $\Delta S = 0$ and $\Delta L = -1$.

\section{ChemActivity 22, Exercise 6}
\begin{enumerate}
  \item 3$d^1$ $\rightarrow$ 4$d^1$: No, $\Delta L = 0$.
  \item 3$d^1$ $\rightarrow$ 4$p^1$: Yes, $\Delta S = 0$ and $\Delta L = -1$.
  \item 3$d^1$ $\rightarrow$ 4$s^1$: No, $\Delta L = -2$.
  \item $^1\Delta$\sub{g} $\rightarrow$ $^1\Delta$\sub{g}: No, forbidden from gerade to gerade.
  \item $^1\Delta$\sub{g} $\rightarrow$ $^1\Pi$\sub{g}: No, forbidden from gerade to gerade.
  \item $^3\Delta$\sub{g} $\rightarrow$ $^3\Delta$\sub{u}: Yes, g $\rightarrow$ u and $\Delta S = 0$.
  \item $^3\Delta$\sub{g} $\rightarrow$ $^1\Delta$\sub{u}: No, $\Delta S = -1$.
\end{enumerate}

\section{Construction of Electrode}
There are a few key components to the electrode they constructed to do their surface-enhanced Raman spectroscopy (SERS). The most essential feature is the size: nanospheres as the layer provides the increased resolution in the absorption of the light being shone on to the TTF-like molecules. Since SERS requires a metal surface to be utilized, it makes sense that gold would be chosen. It has some very good conductive properties (for electrochemistry) which assists in increasing the rate by which these spectroscopic changes happen. The nanospheres provide a similar effect by decreasing the size of the surfaces they interact with. This is how the resolution is able to be so carefully recorded by the electrode. Ultimately, these layers stack on each other from the original gold lead coming in. They might have used titanium in a few layers to help connect the leads up to different surfaces (like the Teflon plug and the spheres). This may have been chosen for its flexibility and low density (but high strength and durability too). Overall, the resolution attained by the nanospheres is sufficient for the researchers' goals for the TTF structure.


\chapter{May 11\super{th}, 2011}

\section{ChemActivity T15, Exercise 4}
Similar to toluene, xylene is simply an additional \ce{CH3} group on the ring of the carbon ring (not attached to the methyl group already attached). A few structural isomers of this molecule do exist. Using the information provided in the problem, a table was constructed like that of Table 1 in this section to represent the composition.
\begin{center}
\begin{tabular}{c|c|c|c|c|c}
  moles of $tol$ & moles of $xyl$ & $X_{tol(sol)}$ & $P_{tol}$ (Torr) & $P_{xyl}$ (Torr) & $P_{tot}$ (Torr)	\\ \hline \hline
  1.0 & 0 & 1 & 25 & 0 & 25 \\ \hline
  0 & 1.0 & 0 & 0 & 5 & 5 \\ \hline
  1.0 & 1.0 & 0.50 & 13 & 3 & 15 \\
  \hline
\end{tabular}
\end{center}
The value of $X_{tol(sol)}$ is defined by the ratio of the moles, and there is an equation true so that $1 - X_{tol(sol)} = X_{xyl(sol)}$. Now for 1.0 moles of each substance, the mole ratio turns out to be $1 / (1 + 1) = 1/2$. The two values for the mole ratios of each substance are used to multiply that by the original pressures of the pure substances not mixed. These two values for toluene and xylene turn out to be 12.5 Torr and 2.5 Torr, respectively. Note the table uses significant figures to round up, but the sum of the two for the total pressure turns out to be 15 Torr. Toluene contributes 83\% to the pressure while xylene contributes 17\% to the total pressure, so there is more toluene vapor in the mixture.

\section{ChemActivity T15, Exercise 6}
Yes, it is possible for both the liquid and vapor phases of a mixture to contain a similar composition. But since there is a conservation of each component of the mixture and each component must be in one phase or the other, there is a limitation on this that each component must be split between the two states. In addition, the relative ratio of the two states must be equal. This means that the ratio of component 1 to component 2 in the liquid phase must be the same as the ratio of those components in the vapor phase. However, the mole ratio defines what the composition split will be, so whatever is not in the vapor phase must be in the liquid phase, or $1 - X_{A(sol)} = X_{A(vap)} = X_{B(sol)}$. There is another equation for component B that when you solve this system of equations, the only solution occurs when all mole ratios are exactly $1/2$. So this will only be true when the components are each split exactly in half between both states.

\section{ChemActivity T16, Exercise 5}
If the equation we found in CTQ8 applies to a non-ideal solution of salt in hot water, then we know that the solution has been changed by the addition of a solute. As we explained in the focus question, the entropy of the solution is increased with the addition of salt (since a mixture is less pure and more mixed up and more ``chaotic''). In addition, we explained that the Gibb's free energy of the solution must decrease with the salt added. This means that the entire solution has a greater change in Gibb's free energy as the solution boils. This increased change in free energy means that it takes more energy for this transition to occur and thus that the boiling point temperature will have increased to allow this phase change to occur with the additional solute added.

\section{ChemActivity T17, Exercise 2}
For the total volume to be less than one of the original species, that means one of the species is ``fitting'' into the volume of the other, or this means that there are spaces in-between molecules of one that allow for an additional species to fit in to, without increasing the overall volume to be simply additive. This could be possible for a partial molar volume turning out to be negative. This is possible. The only limitation for volume mixing is that $\Delta V_\textrm{mix} \neq 0$.

\section{ChemActivity T17, Exercise 3}
For the mixture, it is true that $\mu$\sub{A(sol)} $= \mu^*_{\textrm{A}(l)} + RT \ln{\chi_{\textrm{A}(sol)}}$ and $\mu$\sub{B(sol)} $= \mu^*_{\textrm{B}(l)} + RT \ln{\chi_{\textrm{B}(sol)}}$. Additionally, the worksheets we have received illustrate that for an ideal solution both $\Delta$\sub{mix}$V = 0$ and $\Delta$\sub{mix}$H = 0$. This means that $\bar{H_{\textrm{A}}}$ = -$\bar{H_{\textrm{B}}}$ and $\bar{V_{\textrm{A}}}$ = -$\bar{V_{\textrm{B}}}$. \\ \\
We can also find the entropy for each using the equation for constant pressure and mole ratio: $\bar{S_\textrm{i}} = -\frac{\partial\mu_\textrm{i}}{\partial T}$. And the pure chemical potential is not affected by temperature, so this term cancels out to zero. With that, we can get the two entropies: $\bar{S_\textrm{i}} = -R \ln{\chi_{\textrm{i}(sol)}}$.

\section{Hydrogen Peroxide Reactivity}
Gold does not readily react with hydrogen peroxide like silver does, because the forces between gold particles are stronger than that of silver. This is likely mainly due to the greater intermolecular dispersive forces in gold, since gold and silver have very, very similar electronic structures. The gold atoms about the same size but has more electrons, so more dispersion can occur and hold the gold particles together, which is a stronger force to hold the atoms together than in comparison with the silver. In addition, the oxidation potential plays a critical factor; silver more readily oxidizes from hydrogen peroxide since silver has a less positive potential then gold does. This is why the experimenters were able to easily remove silver and leave gold, since silver will oxidize more readily than gold.


\chapter{May 18\super{th}, 2011}

\section{C\sub{2h} Operations on Unit Vectors}
For the C\sub{2h} point group, we can decompose the cardinal directions of x, y, and z into a representative element by analyzing what each of the operations in the point group does to the axis. Remember that the z-axis is the same as the C\sub{2} axis along with the x and y axes lying in the horizontal plane of reflection of the molecule.
\begin{center}
\begin{tabular}{r|c|c|c|c|}
  & E & C\sub{2} & i & $\sigma$\sub{h} \\ \hline \hline
  \textbf{x} & 1 & -1 & -1 & 1 \\ \hline
  \textbf{y} & 1 & -1 & -1 & 1 \\ \hline
  \textbf{z} & 1 & 1 & -1 & -1 \\
  \hline
\end{tabular}
\end{center}
These values can be taken by simply picturing the three unit vectors at the center of this symmetry and performing their operations, assigning a 1 if they remain the same direction and -1 if they become pointing in the opposite direction. This matches up with the character table's values of B\sub{u}, B\sub{u}, and A\sub{u}, respectively, which also matches up with the column marking the linear parts of the table too.

\section{More on the C\sub{2h} Point Group}
\begin{enumerate}
  \item Each symmetry operation in the point group can be written as a NxN matrix for N-dimensional symmetry. I will assume N=3, the three-dimensional world we experience. The matrices below are all operations on the vector $\vec{v}$. \\ \\
    \begin{align*}
      \textbf{E} = \left[\begin{array}{ccc}
        x & 0 & 0 \\
        0 & y & 0 \\
        0 & 0 & z
      \end{array}\right], 
      \textbf{C\sub{2}} = \left[\begin{array}{ccc}
        -x & 0 & 0 \\
        0 & -y & 0 \\
        0 & 0 & z
      \end{array}\right], 
      \textbf{i} = \left[\begin{array}{ccc}
        -x & 0 & 0 \\
        0 & -y & 0 \\
        0 & 0 & -z
      \end{array}\right], 
      \textbf{$\sigma$\sub{h}} = \left[\begin{array}{ccc}
        x & 0 & 0 \\
        0 & y & 0 \\
        0 & 0 & -z
      \end{array}\right]
    \end{align*}
  \item The components above form $\Gamma$\sub{m}. Using block diagonalization, we can decompose this into component irreducible representations for x, y, and z (by dividing by the respective coordinate).
\begin{center}
\begin{tabular}{r|c|c|c|c|}
  & E & C\sub{2} & i & $\sigma$\sub{h} \\ \hline \hline
  \textbf{C\sub{11}} & 1 & -1 & -1 & 1 \\ \hline
  \textbf{C\sub{22}} & 1 & -1 & -1 & 1 \\ \hline
  \textbf{C\sub{33}} & 1 & 1 & -1 & -1 \\
  \hline
\end{tabular}
\end{center}
    This table is identical to the one established in the previous problem, so we get the following symmetry types: B\sub{u} and A\sub{u}, where B\sub{u} is represented twice.
  \item So we have the following reducible representation for the vector $\vec{v}$:
  \begin{center}
  \begin{tabular}{r|c|c|c|c|}
    & E & C\sub{2} & i & $\sigma$\sub{h} \\ \hline \hline
    $\Gamma$\sub{v} & x+y+z & -(x+y-z) & -(x+y+z) & x+y-z \\
    \hline
  \end{tabular}
  \end{center}
  \item For the type A\sub{g}, the irreducible representation is $\frac{1}{4}(1*1*(x+y+z) + 1*-1*(x+y-z) + 1*-1*(x+y+z) + 1*1*(x+y-z)) = 0$. \\ \\
        For the type B\sub{g}, the irreducible representation is $\frac{1}{4}(1*1*(x+y+z) + -1*-1*(x+y-z) + 1*-1*(x+y+z) + -1*1*(x+y-z)) = 0$. \\ \\
        For the type A\sub{u}, the irreducible representation is $\frac{1}{4}(1*1*(x+y+z) + 1*-1*(x+y-z) + -1*-1*(x+y+z) + -1*1*(x+y-z)) = \frac{4z}{4} = z$. \\ \\
        For the type B\sub{u}, the irreducible representation is $\frac{1}{4}(1*1*(x+y+z) + -1*-1*(x+y-z) + -1*-1*(x+y+z) + 1*1*(x+y-z)) = \frac{4*(x+y)}{4} = x+y$. \\ \\
    This brings us to the same conclusion that B\sub{u} is symmetry for both x \& y and there is also a component of A\sub{u} symmetry in the C\sub{2h} point group using both block diagonalization and matrix traces.
  \item Any multiplication involving the identity operation is simple; the multiplication of an identity matrix \textbf{E} times a matrix \textbf{A} equals simply \textbf{A}. This covers the first row and column entirely of the multiplication table, leaving only the nine other options. Additionally, any operation multiplied by itself must yield the identity matrix \textbf{E}, which is true (any one will remain a one and all negative ones will become positive, leaving an identity matrix). These multiplications are all simplified by the fact that all non-zero terms are along the diagonal, and as such we can also state that \textbf{A}*\textbf{B} = \textbf{B}*\textbf{A}. Thus, we only have three matrix multiplications that we need to show to define a multiplication table.
  \begin{align*}
    \textbf{C\sub{2}}*\textbf{i} = 
    \left[\begin{array}{ccc}
      -1 & 0 & 0 \\
      0 & -1 & 0 \\
      0 & 0 & 1
    \end{array}\right]*
    \left[\begin{array}{ccc}
      -1 & 0 & 0 \\
      0 & -1 & 0 \\
      0 & 0 & -1
    \end{array}\right] = 
    \left[\begin{array}{ccc}
      1 & 0 & 0 \\
      0 & 1 & 0 \\
      0 & 0 & -1
    \end{array}\right] = \textbf{$\sigma$\sub{h}}
  \end{align*}
  \begin{align*}
    \textbf{C\sub{2}}*\textbf{$\sigma$\sub{h}} = 
    \left[\begin{array}{ccc}
      -1 & 0 & 0 \\
      0 & -1 & 0 \\
      0 & 0 & 1
    \end{array}\right]*
    \left[\begin{array}{ccc}
      1 & 0 & 0 \\
      0 & 1 & 0 \\
      0 & 0 & -1
    \end{array}\right] = 
    \left[\begin{array}{ccc}
      -1 & 0 & 0 \\
      0 & -1 & 0 \\
      0 & 0 & -1
    \end{array}\right] = \textbf{i}
  \end{align*}
  \begin{align*}
    \textbf{i}*\textbf{$\sigma$\sub{h}} = 
    \left[\begin{array}{ccc}
      -1 & 0 & 0 \\
      0 & -1 & 0 \\
      0 & 0 & -1
    \end{array}\right]*
    \left[\begin{array}{ccc}
      1 & 0 & 0 \\
      0 & 1 & 0 \\
      0 & 0 & -1
    \end{array}\right] = 
    \left[\begin{array}{ccc}
      -1 & 0 & 0 \\
      0 & -1 & 0 \\
      0 & 0 & 1
    \end{array}\right] = \textbf{C\sub{2}}
  \end{align*}
    With all this information, we can now construct a multiplication table for the C\sub{2h} point group which follows from its reducible representation matrices, $\Gamma$\sub{m}.
 	\begin{center}
	\begin{tabular}{r|c|c|c|c|}
	  & E & C\sub{2} & i & $\sigma$\sub{h} \\ \hline \hline
	  E & E & C\sub{2} & i & $\sigma$\sub{h} \\ \hline
	  C\sub{2} & C\sub{2} & E & $\sigma$\sub{h} & i \\ \hline
	  i & i & $\sigma$\sub{h} & E & C\sub{2} \\ \hline
      $\sigma$\sub{h} & $\sigma$\sub{h} & i & C\sub{2} & E \\
	  \hline
	\end{tabular}
	\end{center}
\end{enumerate}
