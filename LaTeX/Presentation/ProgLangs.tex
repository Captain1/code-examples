% by Sean McKenna

\documentclass{beamer}
\usetheme{Madrid}
\usefonttheme{serif}
\setbeamercovered{invisible}
\setbeamertemplate{navigation symbols}{}
\usepackage{graphicx}
\usepackage{hyperref}
\usepackage{listings}
\lstset{language=Ruby, frame=single, numbers=left, numbersep=-7pt, tabsize=2, keepspaces=true, columns=fullflexible, basicstyle=\ttfamily\scriptsize, keywordstyle=\color{blue}}

\title{Ruby on Rails}
\author{Sean McKenna}
\institute[Cornell College]{ Cornell College \\ \medskip {\emph{smckenna12@cornellcollege.edu}} }
\date{\today}


\begin{document}


\begin{frame}
  \titlepage
\end{frame}

\begin{frame}
  \frametitle{Introduction}
  \begin{itemize}
    \item Rails - web framework using Ruby \medskip
    \item MVC architecture \medskip
    \item specific Ruby gems/packages
    \begin{itemize}
      \item Active Record - DB interaction
      \item Action Pack - displaying info to user in browser
      \begin{itemize}
        \item Action Controller
        \item Action View
      \end{itemize}
    \end{itemize} \medskip
    \item generates specific folder structure \medskip
    \item can create web applications in a workflow \medskip
    \item often combined with version control
  \end{itemize}
\end{frame}

\begin{frame}
  \frametitle{Example - Video Blog}
  \begin{figure}
    \includegraphics[scale=0.3]{list.png}
    \caption{List all video blogs}
  \end{figure}
\end{frame}

\begin{frame}
  \frametitle{Example - Video Blog}
  \begin{figure}
    \includegraphics[scale=0.3]{show.png}
    \caption{Display a single video blog with comments}
  \end{figure}
\end{frame}

\begin{frame}
  \frametitle{Example - Video Blog}
  \begin{figure}
    \includegraphics[scale=0.3]{edit.png}
    \caption{Editing a video blog}
  \end{figure}
\end{frame}

\begin{frame}
  \frametitle{MVC Architecture}
  \begin{figure}
    \includegraphics[scale=1.2]{mvc.png}
    \caption{MVC architecture}
  \end{figure}
  \begin{itemize}
    \item \textbf{Model}: business logic, database (CRUD), Active Record
    \item \textbf{View}: interacts with user, HTML templates, Action View
    \item \textbf{Controller}: model (interact), view (render), Action Controller
  \end{itemize}
\end{frame}

\begin{frame}[fragile]
  \frametitle{Model}
  \begin{block}{Scaffolding}
    rails generate scaffold Video title:string user\_id:integer embed\_code:text
  \end{block}
  \begin{lstlisting}{}
    class Video < ActiveRecord::Base
      attr_accessible :embed_code, :title
      validates :title, :embed_code, :presence => true
      validate :must_have_valid_embed_code

      has_many :comments

      def must_have_valid_embed_code
  	    unless embed_code.include?("<iframe")
  		    errors.add(:embed_code, "Must include an iframe tag")
  	    end
      end
    end
  \end{lstlisting}
\end{frame}

\lstset{language=HTML}
\begin{frame}[fragile]
  \frametitle{View}
  \begin{lstlisting}{}
    <h1>Listing videos</h1>
    <table>
      <tr>
        <th>Title</th>
        <th>Embed code</th>
        <th></th>
        <th></th>
        <th></th>
      </tr>
      <% @videos.each do |video| %>
        <tr>
          <td><%= video.title %></td>
          <td><%= sanitize_embed_code video %></td>
          <td><%= link_to 'Show', video %></td>
          <td><%= link_to 'Edit', edit_video_path(video) %></td>
          <td><%= link_to 'Destroy', video, :confirm => 'Are you sure?',
                          :method => :delete %></td>
        </tr>
      <% end %>
    </table>
    <br />
    <%= link_to 'New Video', new_video_path %>
  \end{lstlisting}
\end{frame}

\lstset{language=Ruby}
\begin{frame}[fragile]
  \frametitle{Controller}
  \begin{lstlisting}{}
    class VideosController < ApplicationController
      def index
        @videos = Video.all
        respond_to do |format|
          format.html # index.html.erb
          format.json { render :json => @videos }
          format.atom
        end
      end

      def show    ...    end
      def new    ...    end
      def edit    ...    end
      def create    ...    end
      def update    ...    end
      def destroy    ...    end

    end
  \end{lstlisting}
\end{frame}

\begin{frame}
  \frametitle{References}
  \begin{itemize}
    \item \url{http://www.sitepoint.com/learn-ruby-on-rails-8/} \bigskip
    \item \url{http://ruby.railstutorial.org/chapters/a-demo-app\#top} \bigskip
    \item \url{http://ofps.oreilly.com/titles/9780596521424/rails.html}
  \end{itemize}
\end{frame}


\end{document}
