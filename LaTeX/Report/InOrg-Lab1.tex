% by Sean McKenna

% This is a LaTeX version of a laboratory report
\documentclass[11pt]{article}

% necessary imports
\usepackage{enumerate}
\usepackage{amsmath}
\usepackage{amssymb}
\usepackage{makeidx}
\usepackage{graphicx}
\usepackage{hyperref}
\usepackage[utf8]{inputenc}
\usepackage{geometry}
\usepackage{booktabs}
\usepackage{array}
\usepackage{verbatim}
\usepackage{subfig}
\usepackage[super]{natbib}
\usepackage{mciteplus}

% page adjustments
\setlength{\parindent}{0pt}
\geometry{a4paper}

% expand margins?
%\addtolength{\oddsidemargin}{-.5in}
%\addtolength{\evensidemargin}{-.5in}
%\addtolength{\textwidth}{1in}
%\addtolength{\topmargin}{-.5in}
%\addtolength{\textheight}{1in}

% double-spaced?
%\linespread{2}

%define commands for super and sub scripts in text
\newcommand{\super}[1]{\ensuremath{^{\textrm{#1}}}}
\newcommand{\sub}[1]{\ensuremath{_{\textrm{#1}}}}

% title
\title{Preparation and Study of a \\
Cobalt(II) Oxygen Adduct Complex}
\author{Sean McKenna}
\date{\today}

% begin the document
\begin{document}
\maketitle

% header info
\begin{center}
\begin{tabular}{lr}
Dates Performed: March 14-15, 2012 & Lab Partner: Adam Jones\\
Instructor: Professor Cindy Strong
\end{tabular}
\end{center}


% describe the experiment's method
\section{Method}
Following the provided procedure,\cite{lab} both salen-H\sub{2} and Co(salen) were produced. \\

During the procedure, the instructions were followed by adding both salicylaldehyde and ethylenediamine into the boiling ethanol, which differed from what was stated in the syllabus. The salen-H\sub{2} product had a 15\% yield which was not enough to continue with the next stage of producing Co(salen). As such, Professor Strong provided additional salen-H\sub{2} produced by past lab students. The melting point for salen-H\sub{2} was measured with only the product from the synthesis done in lab. \\

During part C of the lab, while shaking the side arm test tube with the DMSO and Co(salen) so that the Co(salen) was taking up oxygen, the small test tube punctured the large tube, causing an equilibration of pressure and loss of the brown product through the punctured tube. The measuring buret had been moved by the pressure about 1.97 mL, but the movable tube was not re-adjusted in height for proper measurement of the volume of O\sub{2} gas bound to the Co(salen) complex. Due to safety concerns, focus was shifted to cleaning, so the rest of the lab was lost because the oxygen bound could not be measured due to the leak in the system. \\

The NMR spectrum of salen-H\sub{2} could not be taken since the product had already been used in the synthesis of Co(salen); it was accidentally not set aside for later. \\

The melting and decomposition points were taken with a Mel-Temp Electrothermal.


% show the reaction/s
\section{Reactions}
\begin{center}
2C\sub{7}H\sub{6}O\sub{2} (aq) + C\sub{2}H\sub{8}N\sub{2} (aq) $\rightarrow$ C\sub{16}H\sub{16}N\sub{2}O\sub{2} (s) + 2H\sub{2}O (l) \\
\textit{or} \\
2[salicylaldehyde] + ethylenediamine $\rightarrow$ salen-H\sub{2} + 2H\sub{2}O \bigskip

\textbf{and} \bigskip

C\sub{16}H\sub{16}N\sub{2}O\sub{2} (s) + Co(C\sub{2}H\sub{3}O\sub{2})\sub{2} (s) $\rightarrow$ CoC\sub{16}H\sub{14}N\sub{2}O\sub{2} (s) + 2C\sub{2}H\sub{4}O\sub{2} (aq) \\
\textit{or} \\
salen-H\sub{2} + cobalt(II) acetate $\rightarrow$ Co(salen) + 2[acetic acid]
\end{center}


% show some calculations
\section{Sample Calculations}
\begin{center}
$n = (P - f) \frac{V}{RT} = (101.325 \textrm{kPa} - 3.5681 \textrm{kPa}) \frac{1.97 \textrm{mL} * \frac{1 \textrm{L}}{100 \textrm{mL}}}{8.3145 \frac{\textrm{L kPa}}{\textrm{K mol}} * 300.15 \textrm{K}} = 7.72\times10^{-4} \textrm{moles-O\sub{2}}$
\end{center}


% show any final results from the lab, tabular form
\section{Results}
For both reactions, the following table illustrates the yield and percent yield.

\begin{center}
\begin{tabular}{|l|c|r|}
\hline
\textbf{Product} & \textbf{Yield} & \textbf{\% Yield} \\
\hline
salen-H\sub{2} & 0.082 g & 14.6\% \\
Co(salen) & 0.095 g & 38.5\% \\
\hline
\end{tabular}
\end{center}

For the melting point range of salen-H\sub{2} and the decomposition point of Co(salen), these temperatures are illustrated in the table below.

\begin{center}
\begin{tabular}{|l|r|}
\hline
\textbf{Product} & \textbf{T  (\super{$\circ$}C}) \\
\hline
salen-H\sub{2} & 126 - 127 \\
Co(salen) & $\sim$ 340 \\
\hline
\end{tabular}
\end{center}

The oxygen uptake shown in the Sample Calculations section cannot be used to actually predict the number of moles of O\sub{2} gas taken up by the Co(salen) because the movable tube was not readjusted in height before taking the measurement. This was due to the safety concerns that were more prevalent at the time, but it means that no proper measurement of the uptake was taken and the number shown there is just an example of what the uptake might have been if the reaction had gone to completion.


% discuss what was found, compare to literature, answer specified questions
\section{Discussion}
From a chemical description of salen-H\sub{2},\cite{salenh2} the melting point is 127 - 128\super{$\circ$}C. This is in close agreement with the melting point measured, so it is likely that the first product is mostly the salen-H\sub{2}. \\

According to another lab on oxygen uptake,\cite{cosalen} the decomposition point of Co(salen) is around $\sim$300\super{$\circ$}C. While the value measured differs, it is likely the value listed is meant as a very approximate value when decomposition slowly begins. The value measured was when the product was turning a dark black and very rapidly decomposing. With the similar decomposition points and the proper color for Co(salen), it is likely that the product synthesized was mostly Co(salen).


% answer extra questions
%\section{Questions}


% add the references
\raggedright
\bibliographystyle{achemso}
\bibliography{sources}


% and that's a wrap!
\end{document}
