% by Sean McKenna

% This is a LaTeX version of a laboratory report
\documentclass[11pt]{article}

% necessary imports
\usepackage{enumerate}
\usepackage{amsmath}
\usepackage{amssymb}
\usepackage{makeidx}
\usepackage{graphicx}
\usepackage{hyperref}
\usepackage[utf8]{inputenc}
\usepackage{geometry}
\usepackage{booktabs}
\usepackage{array}
\usepackage{verbatim}
\usepackage{subfig}
\usepackage[super]{natbib}
\usepackage{mciteplus}

% page adjustments
\setlength{\parindent}{0pt}
\geometry{a4paper}

% expand margins?
%\addtolength{\oddsidemargin}{-.5in}
%\addtolength{\evensidemargin}{-.5in}
%\addtolength{\textwidth}{1in}
%\addtolength{\topmargin}{-.5in}
%\addtolength{\textheight}{1in}

% double-spaced?
%\linespread{2}

%define commands for super and sub scripts in text
\newcommand{\super}[1]{\ensuremath{^{\textrm{#1}}}}
\newcommand{\sub}[1]{\ensuremath{_{\textrm{#1}}}}

% title
\title{Preparation and Resolution of Optical Isomers}
\author{Sean McKenna}
\date{\today}

% begin the document
\begin{document}
\maketitle

% header info
\begin{center}
\begin{tabular}{lr}
Dates Performed: March 19 and 21, 2012 & Lab Partner: Laura Kelton \\
Instructor: Professor Cindy Strong
\end{tabular}
\end{center}


% describe the experiment's method
\section{Method}
Using the provided lab procedure,\cite{lab} there was an attempt to carefully isolate and identify the optical isomers of Co(en)\sub{3}\super{3+}. \\

The first section of the procedure was followed, but there was a mistake during the resolution of the cobalt(III) ion. While washing the positive optical isomer, 20mL of half acetone-half water was accidentally poured into the filtrate that should have been separated. It was separated before adding another 20 mL of acetone for washing. \\

The positive isomer's iodide complex was not synthesized because the positive isomer of Co(en)\sub{3} with tartaric acid was not optically active according to measurements. Since an optically inactive complex would not become optically active on its own, this part of the procedure was skipped due to time constraints. However, in an attempt to characterize the isomer, a spectrum was taken on an Agilant 8453 UV-visible spectrophotometer. \\

Using the filtrate separated from the positive optical isomer, there was a significant liquid that was first boiled off to remove as much of the acetone and excess water in the filtrate. Then, following the procedure again, the negative optical isomer was synthesized. The purification step was skipped due to time constraints, so the powdery impure product was collected and tested for optical activation. Similarly, no racemization was tested.


% show the reaction/s
\section{Reactions\cite{lab}}
\begin{center}
CoCl\sub{2}$\cdot$6H\sub{2}O + 3[en$\cdot$2HCl] $\rightarrow$ [Co(en)\sub{3}]Cl\sub{2} + 6H\sub{2}O + 6HCl \\
\textit{then} \\

[Co(en)\sub{3}]Cl\sub{2} + $\frac{1}{2}$NaOH + $\frac{3}{2}$HCl + $\frac{1}{2}$H\sub{2}O\sub{2} + $\frac{3}{2}$H\sub{2}O $\rightarrow$ [Co(en)\sub{3}]Cl\sub{3}$\cdot \frac{1}{2}$NaCl$\cdot$3H\sub{2}O \bigskip

\textbf{and} \bigskip

(+)Co(en)\sub{3}\super{3+} + (+)tart\super{2-} $\rightarrow$ [(+)Co(en)\sub{3}][(+)tart]Cl$\cdot$5H\sub{2}O \\
\textit{and its optical isomer:} \\
(-)Co(en)\sub{3}\super{3+} + (+)tart\super{2-} $\rightarrow$ [(-)Co(en)\sub{3}][(+)tart]Cl \bigskip

\textbf{and} \bigskip

[(+)Co(en)\sub{3}][(+)tart]Cl + 3I\super{-} $\rightarrow$ [(+)Co(en)\sub{3}]I\sub{3}$\cdot$H\sub{2}O + (+)tart\super{2-} + Cl\super{-} \\
\textit{and its optical isomer:} \\

[(-)Co(en)\sub{3}][(+)tart]Cl + 3I\super{-} $\rightarrow$ [(-)Co(en)\sub{3}]I\sub{3}$\cdot$H\sub{2}O + (+)tart\super{2-} + Cl\super{-} \\

\end{center}


% show some calculations
%\section{Sample Calculations}


% show any final results from the lab, tabular form
\section{Results}
Product [Co(en)\sub{3}]Cl\sub{3}$\cdot \frac{1}{2}$NaCl$\cdot$3H\sub{2}O had a yield of 88.7\%. Product [(+)Co(en)\sub{3}][(+)tart]Cl$\cdot$5H\sub{2}O had a yield of 20.2\%. Product [(-)Co(en)\sub{3}]I\sub{3}$\cdot$H\sub{2}O had 4.047 g of wet product. \\

For optical activation, the last two products were measured. For both, the solution of pure water and the products had the same angle of polarization. \\

Finally, for the UV-spectrum of [(+)Co(en)\sub{3}][(+)tart]Cl$\cdot$5H\sub{2}O dissolved in water, the peak wavelengths were found to be 339 nm and 466 nm.


% discuss what was found, compare to literature, answer specified questions
\section{Discussion}
No optical activation was found for either optical isomer. Both the (+)Co with tartaric acid and the (-)Co with iodide showed no change in the polarization of light. \\

However, the UV-spectrum for the (+)Co with tartaric acid was in close agreement with a similar spectrum for (+)Co(en)\sub{3}\super{3+} not in water but in the solvent Co(NH\sub{3})\sub{6}\super{3+}.\cite{spectrum}

\begin{center}
\begin{tabular}{|l|c|c|}
\hline
\textbf{} & \textbf{$\lambda$\sub{A} (nm)} & \textbf{$\lambda$\sub{B} (nm)} \\
\hline
Measured & 339 & 466 \\
Accepted\cite{spectrum} & 342 & 476 \\
\hline
\end{tabular}
\end{center}

While this would support that the product was isolated, it is possible that impurities occurred. For example, there is no steady rise in the peak down to 200 nm as is the case in the accepted spectrum for (+)Co(en)\sub{3}\super{3+} solutions.\cite{spectrum}


% answer leftover questions from the handout
%\section{Questions}


% add the references
\raggedright
\bibliographystyle{achemso}
\bibliography{sources}


% and that's a wrap!
\end{document}
